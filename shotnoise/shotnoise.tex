\documentclass[10pt,a4paper]{article}
\usepackage{graphicx}
\usepackage{amsmath}
\usepackage{dcolumn}
\usepackage{bm}
\usepackage{hyperref}
\usepackage{setspace}
\usepackage[margin=2.5cm]{geometry}
\doublespacing 

\begin{document}

\title{SQUID as a shot noise thermometer.}
\maketitle

\begin{abstract}
A Squid at the end of a transmission line is used as a Shot Noise Thermometry.
This is done by applying a DC current larger than the critical current  across the squid.
For larger currents, the resistance of the squid becomes linear.
The squid can then be used as a shot noise sample to calibrate the gain and the noise temperature 
of the circuit.
\end{abstract}

\begin{eqnarray}
S_I (f, V, T, R) &=&  \frac{2k_BT}{R}  
\left[
\frac{eV + hf}{2k_BT} 
coth \left( \frac{eV + hf}{2k_BT} \right) +
\frac{eV - hf}{2k_BT} 
coth \left( \frac{eV - hf}{2k_BT} \right) 
\right]
\end{eqnarray}

For a digitizer BW < 
\begin{eqnarray}
P(f, V, T, R) &=& GBW 
\left[ 
k_B Tn(f) + 
\left(
\frac{Z_0}{Z_0 + R}
\right)^2
S_I(f, V, T, R)
\right]
\end{eqnarray}

\begin{eqnarray}
Tn(f) &=&
\left[ 

\right]
\end{eqnarray}

\end{document}